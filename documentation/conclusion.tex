\setlength{\footskip}{8mm}

\chapter{Conclusion}
\label{ch:conclusion}

% \textit{This chapter provides a summary of the findings listed in the previous chapter and their implications. Further development of the proposed system is also put forward.}

% \section{Conclusion}
In conclusion, the system developed in this study is successful in facilitating communication between the Pixhawk and a web application through a Raspberry Pi. Through its use, mission automation is automated, with the user only needing to specify the core mission parameters. The remote aspect is handled by using an SSH tunnel.

From the total flight time, there is no significant difference between using this system and a locally connected simulator drone on the webserver. However, the mission success rate as found in the results is only approximately 60\%, implying that further hardware optimization and development is necessary. The main source of error is the hardware on the drone. Finally, the accuracy of the GPS-assisted landing built into the Pixhawk is not sufficient enough for precision landing. Additional modifications such as image processing need to be added to enable this feature and utilize the base to its maximum potential.

% total time no differ
% accu

\FloatBarrier

